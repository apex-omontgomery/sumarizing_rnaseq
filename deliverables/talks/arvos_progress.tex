%% beamer packages
% other themes: AnnArbor, Antibes, Bergen, Berkeley, Berlin, Boadilla, boxes, 
% CambridgeUS, Darmstadt, Dresden, Frankfurt, Goettingen, Hannover, Ilmenau,
%JuanLesPins, Luebeck, Madrid, Malmoe, Marburg, Montpellier, PaloAlto,
%Pittsburgh, Rochester, Singapore, Szeged, Warsaw
% other colors: albatross, beaver, crane, default, dolphin, dove, fly, lily, 
%orchid, rose, seagull, seahorse, sidebartab, structure, whale, wolverine,
%beetle

%\documentclass[xcolor=dvipsnames]{beamer}
\documentclass[table,dvipsnames]{beamer}
\usepackage{beamerthemesplit}
\usepackage{bm,amsmath,marvosym}
\usepackage{listings,color}%xcolor
\usepackage[ngerman]{babel}
\usepackage{natbib}
\usepackage[utf8]{inputenc}
\definecolor{shadecolor}{rgb}{.9, .9, .9}
\definecolor{darkblue}{rgb}{0.0,0.0,0.5}
\definecolor{myorange}{cmyk}{0,0.7,1,0}
%\definecolor{mypurple}{cmyk}{0.5,1.0,0,0.2}
\definecolor{mypurple}{cmyk}{0.3, 0.9, 0.0, 0.2}

% make a checkmark
\usepackage{tikz}
\def\checkmark{\tikz\fill[scale=0.4](0,.35) -- (.25,0) -- (1,.7) -- (.25,.15) -- cycle;} 

% dot product
\usetikzlibrary{arrows,positioning}
\tikzset{
    %Define standard arrow tip
    >=stealth',
    % Define arrow style
    pil/.style={->,thick}
}

% math stuff
\newcommand{\argmin}{\operatornamewithlimits{argmin}}

\lstnewenvironment{code}{
    \lstset{backgroundcolor=\color{shadecolor},
        showstringspaces=false,
        language=python,
        frame=single,
        framerule=0pt,
        keepspaces=true,
        breaklines=true,
        basicstyle=\ttfamily,
        keywordstyle=\bfseries,
        basicstyle=\ttfamily\scriptsize,
        keywordstyle=\color{blue}\ttfamily,
        stringstyle=\color{red}\ttfamily,
        commentstyle=\color{green}\ttfamily,
        columns=fullflexible
    }
}{}

\lstnewenvironment{codeout}{
    \lstset{backgroundcolor=\color{shadecolor},
        frame=single,
        framerule=0pt,
        breaklines=true,
        basicstyle=\ttfamily\scriptsize,
        columns=fullflexible
    }
}{}

\hypersetup{colorlinks = true, linkcolor=darkblue, citecolor=darkblue,urlcolor=darkblue}
\hypersetup{pdfauthor={A. Richards}, pdftitle={Support Vector Machines}}

\newcommand{\rd}{\textcolor{red}}
\newcommand{\grn}{\textcolor{green}}
\newcommand{\keywd}{\textcolor{myorange}}
\newcommand{\highlt}{\textcolor{darkblue}}
\newcommand{\norm}[1]{\left\lVert#1\right\rVert}
\def\ci{\perp\!\!\!\perp}
% set beamer theme and color
\usetheme{Frankfurt}
%\usetheme{Berkeley}
%\usecolortheme{dolphin}
\usecolortheme{seagull}
%\setbeamertemplate{blocks}[rounded][shadow=true]

\title[ARVoS]{Adaptive results visualization of sequences \\ ARVoS}
\author[]{William Montgomery\\ Gareth Halladay \\ Anela Tosevska \\ Frank Burkholder \\ Adam Richards \\ Andrew Gaines}
\date[]{May 2017}

%%%%%%%%%%%%%%%%%%%%%%%%%%%%%%%%%%%%%%%%%%%%%%%%%%%%%%%%%%%%%%%%%%%%%%%%%%%%%%%
\begin{document}
\frame{\titlepage}
% %%%%%%%%%%%%%%%%%%%%%%%%%%%%%%%%%%%%%%%%%%%%%%%%%%%%%%%%%%%%%%%%%%%%%%%%%%%%%%%
% \frame{
% \footnotesize
% \tableofcontents
% \normalsize
% }

%%%%%%%%%%%%%%%%%%%%%%%%%%%%%%%%%%%%%%%%%%%%%%%%%%%%%%%%%%%%%%%%%%%%%%%%%%%%%%%
\frame{
\frametitle{What is ARVoS?}
\footnotesize

Adaptive results visualization of sequences (ARVoS)
\vspace{1cm}
\begin{block}{Motivation}
A dockerized database and flask template for presentation of RNAseq results
\end{block}
\vspace{1cm}
\href{https://github.com/NCBI-Hackathons/arvos}{https://github.com/NCBI-Hackathons/arvos}
}

%%%%%%%%%%%%%%%%%%%%%%%%%%%%%%%%%%%%%%%%%%%%%%%%%%%%%%%%%%%%%%%%%%%%%%%%%%%%%%%
\frame{
\frametitle{Principal Objectives}
\footnotesize
\begin{block}{Motivation}
 \begin{enumerate}
  \item Create a dynamic and interactive results display environment (Flask) (Anela,Frank,William)
  \item Create an environment that encourages model comparison (Gareth,Adam)
  \item Dockerize (Andrew)
 \end{enumerate}
\end{block}
}

%%%%%%%%%%%%%%%%%%%%%%%%%%%%%%%%%%%%%%%%%%%%%%%%%%%%%%%%%%%%%%%%%%%%%%%%%%%%%%%
\begin{frame}[fragile]
\frametitle{Node and edge attributes}
\begin{code}

from arvos import Pipeline

results_dir = os.path.join(".","results")
pline = Pipeline(results_dir)
countsPath = os.path.join(parentDir,"data", "est_counts.csv")
filteredCountsPath = pline.create_filtered(countsPath)
pline.run_deseq(filteredCountsPath,outFile)

deseq_file = os.path.join(results_dir,"deseq.csv")
deseq_matrix_file = os.path.join(results_dir,"deseq-samples.csv")
targets_file = os.path.join(".","data","targets.csv")

X,y = pline.generate_features_and_targets(deseq_file,deseq_matrix_file,targets_file)

\end{code}
\end{frame}


%%%%%%%%%%%%%%%%%%%%%%%%%%%%%%%%%%%%%%%%%%%%%%%%%%%%%%%%%%%%%%%%%%%%%%%%%%%%%%%
\frame{
\frametitle{Demo}
\scriptsize
\href{http://54.213.27.230}{http://54.213.27.230}
}

%%%%%%%%%%%%%%%%%%%%%%%%%%%%%%%%%%%%%%%%%%%%%%%%%%%%%%%%%%%%%%%%%%%%%%%%%%%%%%%
\frame{
\frametitle{Where do we go from here?}
\scriptsize
\begin{itemize}
 \item More interactive plots for RNA-Seq
 \item Better Generalize for results versions
 \item Finish Pieris (Manuscript Supplement)
 \item Finish the Asthma (Manuscript Supplement)
 \item Docs in the style of PyMC3
 \item Blog post
 \item Publish as an application
\end{itemize}
}

\end{document}